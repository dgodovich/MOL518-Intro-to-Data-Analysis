\documentclass[11pt]{article}

\usepackage[margin=1in]{geometry}
\usepackage{amsmath, amssymb}
\usepackage{graphicx}
\usepackage{hyperref}
\usepackage{enumitem}
\usepackage{float}
\usepackage{booktabs}
\usepackage{xcolor}
\usepackage{listings}

\hypersetup{colorlinks=true, linkcolor=blue, urlcolor=blue, citecolor=blue}

% Simple code listing style
\lstset{
  basicstyle=\ttfamily\small,
  frame=single,
  breaklines=true,
  showstringspaces=false,
  keywordstyle=\color{blue},
  commentstyle=\color{gray},
  stringstyle=\color{teal}
}

\newcommand{\course}{MOL518 Spring 2026}
\newcommand{\hwnum}{Homework 2}
\newcommand{\duedate}{\textbf{Due: Friday February 13, 2026}}

\begin{document}

\begin{center}
{\Large \course}\\[2pt]
{\large \hwnum}\\[6pt]
\duedate\\[10pt]
\end{center}


For this assignment, you will generate and then submit a Jupyter notebook on Canvas. You may use generative AI tools as per the syllabus unless otherwise stated in the problem.

\vskip 4pt

\noindent \textit{First, make a new Jupyter notebook named \texttt{MOL518\_HW2\_<YourLastName>.ipynb}. All problems will be answered in this notebook. For each problem, start with a Markdown cell using Problem XX as the title, i.e. ``\# Problem 1''.}


\section*{Problem 1: Looping Through Subplots}

This problem uses the file \texttt{ecoli\_drugs.csv} which we used in Lecture 3. This file has time in the first column and six drug growth curves in the next six columns. {\it Do not use AI to generate code for this problem. You may use AI to debug if you get an error or to help understand why your code isn't working properly.}

\begin{enumerate}[label=\alph*)]
  \item Load the CSV file.
  \item Create a list of the six OD arrays in the following order: Rifampicin, Novabiocin, Trimethoprim, Chloramphenicol, Ampicillin, Gentamycin.
  \item Create a matching list of drug names.
  \item Use \texttt{plt.subplots} to make a 3x2 grid. In class we filled the grid by brute force. Here, fill the panels using \textbf{for loops}, i.e. no manual \texttt{axes[0,0]} indexing. Each panel should include a title and axis labels.
\end{enumerate}

\section*{Problem 2: Mystery Data}

This problem uses the file \texttt{mystery.csv}. {\it Do not use AI to generate code for this problem. You may use AI to debug if you get an error or to help understand why your code isn't working properly.}

\begin{enumerate}[label=\alph*)]
  \item First, inspect the csv file. Then load the file and put the data into two 1D arrays called \texttt{x} and \texttt{y}.
  \item Make a scatter plot of \texttt{y} vs \texttt{x}.
  \item Count how many points have \texttt{y > 5} and display the answer.
  \item Now, repeat this counting for thresholds of 5, 10, 15, 20, and 25. Store the results in two lists: \texttt{thresholds} and \texttt{n\_above}.
  \item Make a line plot of \texttt{n\_above} vs \texttt{thresholds}.
  \item Save a CSV file called \texttt{mystery\_summary.csv} with two columns: \texttt{threshold} and \texttt{n\_above}.
\end{enumerate}

\section*{Problem 3: Summary of Microscopy Images}

In the \texttt{images} folder there are 20 microscopy \texttt{.tif} files of fields of fluorescent bacteria. {\it Do not use AI to generate code for this problem. This problem has some things we only touched on, you may use AI to help solve the problem but make sure to follow the guidelines in the syllabus.}

\begin{enumerate}[label=\alph*)]
  \item Use \texttt{pathlib.Path} to list all \texttt{.tif} files in the folder and print the total count.
  \item Print the names of the first 5 files (alphabetical order).
  \item For each filename, split the stem on underscores (\texttt{\_}) and extract the strain (the second field, e.g. \texttt{Ecoli} in \texttt{2026-02-05_Ecoli_FtsZ-GFP_ctrl_FOV01.tif}).
  \item Use a dictionary to count how many images belong to each cell line, then print the dictionary.
\end{enumerate}

\section*{Problem 4: The Effect of Bin Size}

This problem uses the file \texttt{SOX2\_expression.csv} which contains a column of expression values. {\it Do not use AI to generate code for this problem. You may use AI to debug if you get an error or to help understand why your code isn't working properly.}
\begin{enumerate}[label=\alph*)]
  \item Load the expression values into an array.
  \item Make a histogram with 8 bins.
  \item Create a 1x3 grid of histograms for bin counts \texttt{[8, 20, 40]}. Use \texttt{plt.subplots} and a \texttt{for} loop to fill the panels.
  \item In 2--3 sentences, explain how the choice of bin size changes what you perceive about the distribution.
\end{enumerate}

\section*{Problem 5: Replicate Growth Curves and the Mean}

This problem uses the file  \texttt{ecoli\_growth\_replicates.csv}. The first column is time (hours) and the next 10 columns is OD from replicate experiments. {\it Do not use AI to generate code for this problem. You may use AI to debug if you get an error or to help understand why your code isn't working properly.}

\begin{enumerate}[label=\alph*)]
  \item Load the CSV file into a 2D NumPy array. Extract \texttt{time} and a 2D array \texttt{od\_reps}.
  \item Plot all 10 replicate curves on the same axes in different colors.
  \item Compute the mean OD at each timepoint (average across replicates). Plot the mean OD as a thick black line on the same figure.
  \item Save a new CSV file called \texttt{ecoli\_growth\_mean.csv} with two columns: \texttt{time\_hr} and \texttt{od\_mean}.
\end{enumerate}


\end{document}
