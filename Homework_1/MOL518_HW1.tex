\documentclass[11pt]{article}

\usepackage[margin=1in]{geometry}
\usepackage{amsmath, amssymb}
\usepackage{graphicx}
\usepackage{hyperref}
\usepackage{enumitem}
\usepackage{float}
\usepackage{booktabs}
\usepackage{xcolor}
\usepackage{listings}

\hypersetup{colorlinks=true, linkcolor=blue, urlcolor=blue, citecolor=blue}

% Simple code listing style
\lstset{
  basicstyle=\ttfamily\small,
  frame=single,
  breaklines=true,
  showstringspaces=false,
  keywordstyle=\color{blue},
  commentstyle=\color{gray},
  stringstyle=\color{teal}
}

\newcommand{\course}{MOL518 Spring 2026}
\newcommand{\hwnum}{Homework 1}
\newcommand{\duedate}{\textbf{Due: Friday February 6, 2026}}

\begin{document}

\begin{center}
{\Large \course}\\[2pt]
{\large \hwnum}\\[6pt]
\duedate\\[10pt]
\end{center}


For this assignment, you will generate and then submit a jupyter notebook on canvas. You may use generative AI tools as per the syllabus unless otherwise stated in the problem.

\vskip 4pt

\noindent \textit{First, make a new Jupyter notebook named \texttt{MOL518\_HW1\_<YourLastName>.ipynb}. All problems will be answered in this notebook. For each problem, start with a Markdown cell using Problem XX as the title, i.e. ``\# Problem 1''.}

\newpage 

\section*{Problem 1: Practice with Markdown}

Figure~\ref{fig:stryerpreface} is the preface page for Lubert Stryer's classic textbook \emph{Biochemistry}. Recreate this page in a Markdown cell in your notebook. The image at the bottom of the page is provided as \texttt{StryerPreface\_liverketone.png}. {\it Do not use AI for this problem.}

\begin{figure}[H]
\centering
\includegraphics[width=0.6\textwidth]{StryerPreface.png}
\caption{Stryer preface page.}
\label{fig:stryerpreface}
\end{figure}

\section*{Problem 2: Wet-Lab Arithmetic --- MOI Setup (No Loops)}
(15 points)

You are infecting cells in a single well.
\textbf{Do not use \texttt{for} loops or \texttt{if} statements in this problem.}

\begin{enumerate}[label=\alph*)]
  \item Define the following variables (with these exact names) in Python:
  \begin{itemize}
    \item \texttt{cells} = 250000 \quad (cells)
    \item \texttt{moi} = 0.2 \quad (dimensionless)
    \item \texttt{titer} = 2.0e8 \quad (infectious units per mL)
    \item \texttt{inoc\_vol\_uL} = 500 \quad ($\mu$L total inoculation volume)
  \end{itemize}

  \item Compute the number of infectious units needed for this MOI:
  \[\texttt{iu\_needed} = \texttt{cells} \times \texttt{moi}.\]

  \item Convert the titer to infectious units per $\mu$L, and compute the volume of virus stock to add (in $\mu$L).

  \item Compute how many $\mu$L of media to add so the final inoculation volume is \texttt{inoc\_vol\_uL}.

  \item Print a short, readable summary with your final values (infectious units, virus volume, media volume).
\end{enumerate}

\section*{Problem 3: CSV ``Sanity Check'' Report (No Loops)}
(15 points)

Load growth curve data and compute a few quick summary quantities.
\textbf{Do not use \texttt{for} loops or \texttt{if} statements in this problem.}

\begin{enumerate}[label=\alph*)]
  \item Load \texttt{Lecture\_2/data/growth\_curve1.csv} as a 2D NumPy array called \texttt{data}.
  \item Extract \texttt{time} (seconds) and \texttt{od} (unitless) as 1D arrays.
  \item Print the shape of \texttt{data}, plus the first row and last row.
  \item Compute and print the sampling interval in seconds: \texttt{dt\_sec = time[1] - time[0]}.
  \item Compute and print the total duration in hours.
  \item Compute and print:
  \begin{itemize}
    \item absolute OD change: \texttt{od[-1] - od[0]}
    \item fold change in OD: \texttt{od[-1] / od[0]}
  \end{itemize}
\end{enumerate}

\section*{Problem 4: Create, Transform, and Save a Synthetic Dataset (No Loops)}
(15 points)

You will generate a tiny synthetic ``growth'' dataset (time vs. OD), then save and reload it.
\textbf{Do not use \texttt{for} loops or \texttt{if} statements in this problem.}

\begin{enumerate}[label=\alph*)]
  \item Create a NumPy array called \texttt{time\_hr} with these time points (hours): \texttt{[0, 1, 2, 4, 8, 16]}.
  \item Define \texttt{od0 = 0.08} and \texttt{doubling\_time\_hr = 1.5}.
  \item Compute \texttt{od} using: \texttt{od = od0 * 2**(time\_hr/doubling\_time\_hr)}.
  \item Create \texttt{od\_norm} by normalizing \texttt{od} to its first value.
  \item Use \texttt{np.column\_stack} to make a 2D array with columns \texttt{time\_hr}, \texttt{od}, \texttt{od\_norm}.
  \item Save the result to \texttt{Homework\_1/simulated\_growth.csv} using \texttt{np.savetxt(..., delimiter=",")}.
  \item Reload the file you saved into \texttt{data2} and print \texttt{data2.shape} and the first 3 rows.
\end{enumerate}

\section*{Problem 5: Lists --- Sample Tracking and Lane Batching (No Loops)}
(15 points)

You will practice list creation, indexing, slicing, appending, and sorting.
\textbf{Do not use \texttt{for} loops or \texttt{if} statements in this problem.}

\begin{enumerate}[label=\alph*)]
  \item Create a Python list called \texttt{samples} with the following sample IDs (strings), in this exact order:

\begin{lstlisting}[language=Python]
samples = [
    "WT_A", "WT_B", "WT_C", "KO1_A", "KO1_B", "KO1_C",
    "KO2_A", "KO2_B", "KO2_C", "Rescue_A", "Rescue_B", "Rescue_C"
]
\end{lstlisting}

  \item Print the number of samples using \texttt{len(samples)}.
  \item Append two new samples to the end of the list: \texttt{"Blank"} and \texttt{"PositiveCtrl"}.
  \item Create a new list called \texttt{samples\_sorted} that is an alphabetically sorted version of \texttt{samples} (do not change the original list).
  \item Create three new lists by slicing \texttt{samples\_sorted}:
  \begin{itemize}
    \item \texttt{lane1} = first 5 samples
    \item \texttt{lane2} = next 5 samples
    \item \texttt{lane3} = remaining samples
  \end{itemize}
  \item Print \texttt{lane1}, \texttt{lane2}, and \texttt{lane3}.
\end{enumerate}




\end{document}
