\documentclass[11pt]{article}

\usepackage[margin=1in]{geometry}
\usepackage{amsmath, amssymb}
\usepackage{graphicx}
\usepackage{hyperref}
\usepackage{enumitem}
\usepackage{float}
\usepackage{booktabs}
\usepackage{xcolor}
\usepackage{listings}

\hypersetup{colorlinks=true, linkcolor=blue, urlcolor=blue, citecolor=blue}

% Simple code listing style
\lstset{
  basicstyle=\ttfamily\small,
  frame=single,
  breaklines=true,
  showstringspaces=false,
  keywordstyle=\color{blue},
  commentstyle=\color{gray},
  stringstyle=\color{teal}
}

\newcommand{\course}{MOL518 Spring 2026}
\newcommand{\hwnum}{Homework 1}
\newcommand{\duedate}{\textbf{Due: Friday February 6, 2026}}

\begin{document}

\begin{center}
{\Large \course}\\[2pt]
{\large \hwnum}\\[6pt]
\duedate\\[10pt]
\end{center}


For this assignment, you will generate and then submit a jupyter notebook on canvas. You may use generative AI tools as per the syllabus unless otherwise stated in the problem.

\vskip 4pt

\noindent \textit{First, make a new Jupyter notebook named \texttt{MOL518\_HW1\_<YourLastName>.ipynb}. All problems will be answered in this notebook. For each problem, start with a Markdown cell using Problem XX as the title, i.e. ``\# Problem 1''.}

\newpage 

\section*{Problem 1: Practice with Markdown}

Figure~\ref{fig:stryerpreface} is the preface page for Lubert Stryer's classic textbook \emph{Biochemistry}. Recreate this page in a Markdown cell in your notebook. The image at the bottom of the page is provided as \texttt{StryerPreface\_liverketone.png}. {\it Do not use AI for this problem.}

\begin{figure}[H]
\centering
\includegraphics[width=0.6\textwidth]{StryerPreface.png}
\caption{Stryer preface page.}
\label{fig:stryerpreface}
\end{figure}


\section*{Problem 2: Inspecting a growth curve}


\section*{Problem 3: Saving Analyzed Data}




\end{document}
